
\newglossaryentry{har}{
  name=HAR,
  description = {Human Activity Recognition}
}

\newglossaryentry{gsc}{
    name=GSC,
    description = {Goal-Setting Component}
}

\newglossaryentry{mc}{
    name=MC,
    description = {Monitoring Component}
}

\newglossaryentry{fc}{
    name=FC,
    description = {Feedback Component}
}

\newglossaryentry{pa}{
    name=PA,
    description = {Physical Activity}
}

\newglossaryentry{st}{
    name=ST,
    description = {Sedentary Time}
}

\newglossaryentry{sb}{
    name=SB,
    description = {Sedentary Behaviour}
}

\newglossaryentry{mvp}{
    name=MVP,
    description = {Model View Controller is a design pattern that separates the software logic into tree roles: model, view, or controller}
}

\newglossaryentry{ui}{
    name=UI,
    description = {User Interface}
}

\newglossaryentry{sa}{
    name=SA,
    description = {System Architecture}
}

\newglossaryentry{ide}{
    name=IDE,
    description = {Integrated Development Environment}
}

\newglossaryentry{fft}{
    name=FFT,
    description = {Fast Fourier Transform is an algorithm that converts a signal from its original domain (often time) to a representation in the frequency domain}
}

\newglossaryentry{mvp}{
    name=MVP,
    description = {Model-View-Presenter is software architectural pattern that is a pattern which provides a cleaner separation between the View, the Model and the Presenter}
}

\newglossaryentry{cfs}{
    name=CFS,
    description = {Correlation based Feature Selection}
}

\newglossaryentry{knn}{
    name=kNN,
    description = {k-Nearest Neighbors}
}

\newglossaryentry{oms}{
    name=OMS,
    description = {Object-Oriented Data Management System}
}
\glsaddall %Print all entries %
\setglossarystyle{altlist}
\printglossaries