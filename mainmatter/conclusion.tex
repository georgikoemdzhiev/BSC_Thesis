\chapter{Conclusion}
To sum up, the project initially set seven objectives and achieved the high-level project aim. That is to develop a system that allows for the self-management of \gls{sb} and \gls{pa}. In addition, the system provides the user with goal-performance feedback. 

The purpose of this chapter is to present the main findings discovered during the project development. Furthermore, a short summary of status of the project objectives will be presented followed by a discussion of the future improvements of the system. Finally, a personal reflection regarding the project experience will be covered. 

\section{\textit{ActiveMinutes}}
The project successfully achieved initially set high-level aim and produced a system that applies self-management for both physical states, namely \textit{Active} and \textit{Static}. The system successfully applied behaviour change via three stages, namely goal-setting, monitoring and providing user feedback. The aim of the project was achieved by archiving the following project objectives:

\begin{itemize}
    \item \textit{Research the effect of SB as well as investigate behaviour change approaches}\\
    This objective has been successfully met by investigating the current literature regarding the behaviour change methods and moderators (i.e. what affects behaviour change).
    \item \textit{Investigate and gather information about HAR on wearable devices}\\
    This objective was achieved by analysing current and past \gls{har} systems. For example, recommended classifiers on mobile devices, sensor sampling rate and window size.
    \item \textit{Develop a fully working HAR system}\\
    This objective has been achieved by essentially implementing the research findings from the above-mentioned objective.
    \item \textit{Incorporate the above within an SB self-management system}\\
    When the \gls{har} system was implemented, it was paired with a self-management system that facilitates set-setting, monitoring and feedback functionality
    \item \textit{Design and Implement a mobile application based on the project research findings}\\
    In order to satisfy this objective successfully, a software architecture design was devised in addition to the \gls{ui} design of the mobile application.
    
    \item \textit{Evaluate the system performance using user feedback and analysing the gathered data}\\
    A series of tests, namely \textit{Acceptance};\textit{Functional} and \textit{Usability} have been carried away in addition to carrying a user survey needed to satisfy this objective successfully.
    
    \item \textit{Release app to application to the Application Store}\\
    The mobile application was released to the public in the form of a pre-release version (also called BETA). This way users who want to try the system can do so by downloading the mobile application from Google's Play Store\footnote{\url{https://play.google.com/store/apps/details?id=georgikoemdzhiev.activeminutes}}.
    
\end{itemize}

\section{Future versions}
Although \textit{ActiveMinutes} implements the key requirements needed by a self-management system (e.g. goal-setting, monitoring and user feedback) there are features that can be added in future iterations of the system. 

One of the features that may be addressed in future versions of the system is classifier personalisation. When implemented, it is expected that the system would improve its classification accuracy since the classifier would adapt to the user's specific motor patterns (i.e. walking). Furthermore, personalisation could be applied to the notification content. For example, the system may keep track of what messages have proven to be successfully (e.g. made the user more active) so that only the selected message content is shown to the user.

One of the features that was out of the scope if this project was social media sharing. Future versions of the application could incorporate a sharing feature allowing for sharing content on social media platforms such as \textit{Facebook} and \textit{Twitter} such as the set \gls{pa} goal. As it was found in the literature review (see section \ref{subsection:goal-moderators}) making a goal public increases how important is the goal in ones eyes and that is expected to produce better goal-performance results. 

The feedback gathered via the survey discussed in Chapter \ref{chapter:testing-and-eval} helped to identify other future improvements to the system. For example, integrating a sync feature with a smartwatch platform such as \textit{Android Wear}\footnote{\url{https://www.android.com/intl/en_uk/wear/}} would enable the user to self-manage their activity levels even without a smartphone. That would be particularly more useful when doing activities such as swimming. Furthermore, optimising the system in terms of battery-consumption would be another future improvement. That could be done by experimenting with lower sampling rates of the accelerometer sensor. The user feedback gathered by the survey also suggests improving the \textit{History} screen \gls{ui} so that it is easier to understand what each of the two progress bars mean. That could be improved by adding labels in the progress bars themselves so it is clearer which one shows \gls{pa} and which one shows \gls{mci} progress.

\section{Reflection}
In general, I am satisfied with the project development and the effort I have put into the project research and development. I have worked to the best of my abilities to satisfy the (BSc) Honours Computing (Graphics and Animation) degree requirements.

As far as the project time management is concerned, I have spent approximately two-thirds of my time on the design and research stage. Firstly, research allowed me to gather all of the required background information in order to complete the software implementation of the project. Secondly, the mobile application design ensured that the system implements the core specifications required by a typical self-management system and at the same time make sure that the system is feasible to implement given the project time constraint.

This project allowed me to apply the university knowledge gathered during my four years of study. Also, I was able to practice and improve my industry mobile application development experience gained with xDesign. Going into the project, I was curious about mobile development and how it can be used to help people in their everyday life. Now that the project is complete, I feel confident that mobile applications can certainly be used to support people, more specifically, to help inactive people become more active via the use of mobile systems.
