\chapter{Concluding Remarks}
\label{chapter:conclusion}
To sum up, the project initially set seven objectives which help to achieve the high-level project aim. That is to develop a system that allows the user to self-manage \gls{sb} and \gls{pa} and provide goal-performance feedback in the form of notifications. The purpose of this chapter is to present the main findings discovered during the project development. Furthermore, a short summary of status of the project objectives will be presented followed by a discussion of the potential future improvements of the system. Finally, a personal reflection regarding the project experience will be discussed. 

\section{Objectives achieved}
The project successfully achieved the initially set high-level aim and produced a system that applies self-management for both physical states, namely \textit{Active} and \textit{Static} and it was published on the application store. The system successfully applies behaviour change via three stages - goal-setting, monitoring and providing user feedback. The aim of the project was achieved by attaining the following project objectives:

\begin{itemize}
    \item \textit{Research the effect of \gls{sb} as well as investigate behaviour change approaches}\\
    In order to do this, I have investigated the current literature regarding the behaviour change methods and moderators (i.e. what affects behaviour change, see section \ref{subsection:goal-moderators}).
    \item \textit{Investigate and gather information about \gls{har} on wearable devices}\\
    This was done by analysing current and past \gls{har} systems. For example, the literature review suggested that the optimal algorithm to train a classifier on a mobile system is k-Nearest-Neighbour. Furthermore, the optimal window size was found to be 3 second window length and the accelerometer sampling frequency 128 Hz (see section \ref{section:learning_alg_accuracy_power_consm}).
    \item \textit{Develop a fully working HAR system}\\
    This objective has been achieved by implementing the research findings from the above-mentioned objective and developing a software component (i.e.\ \textit{HarManager}) integrated as part of \textit{ActiveMinutes} that is responsible for all of the \gls{har} related stages such as data reprocessing \& windowing; feature extraction \& selection and classification of the user activity (see section \ref{subsection:har-system-implementation}).
    
    \item \textit{Incorporate the above within an \gls{sb} self-management system}\\
    When the \gls{har} system was implemented, it was paired with a self-management system (see section \ref{section:self-management-system}) that facilitates goal-setting, monitoring and feedback functionality.
    \item \textit{Design and Implement a mobile application based on the project research findings}\\
    In order to satisfy this objective successfully, a software architecture design was devised in addition to the \gls{ui} design of the mobile application.
    
    \item \textit{Evaluate the system performance using user feedback and analysing the gathered data}\\
    A series of tests, namely \textit{Acceptance}; \textit{Functional} and \textit{Usability} have been carried out to evaluate if the system complies with the client requirements, works as expected and is easy to use. In addition, a user survey has been conducted to gather real-user feedback regarding the functionality of the system in general.
    
    \item \textit{Release app to application to the Application Store}\\
    The mobile application was released to the public in the form of a pre-release version (also known as BETA) so users who want to try the system can do so by downloading the mobile application from Google Play Store\footnote{\url{https://play.google.com/store/apps/details?id=georgikoemdzhiev.activeminutes}}.
    
\end{itemize}

\section{Future versions}
Although \textit{ActiveMinutes} implements the key requirements needed by a self-management system (e.g. goal-setting, monitoring and user feedback) there are features that can be added in future iterations of the system. One of the features that may be addressed is classifier personalisation. When implemented, it is expected that the system would improve its classification accuracy since the classifier would adapt to the user's specific motor patterns (i.e. walking). Furthermore, personalisation could be applied to the notification content. For example, the system may keep track of what messages have proven to be successfully (e.g. made the user more active) so that only the selected message content is shown to the user.

One of the features that were not implemented due to the time constraint of the project was social media sharing. Future versions of the application could incorporate such feature allowing for sharing content on social media platforms such as \textit{Facebook} and \textit{Twitter}. For example, the user may share their achieved \gls{pa} or \gls{sb} goal. As it was found in the literature review (see section \ref{subsection:goal-moderators}) making a goal public increases the importance of the goal in ones eyes and that is expected to produce better goal-performance results. 

The feedback gathered via the survey discussed in Chapter \ref{chapter:testing-and-eval} helped to identify other future improvements to the system. For example, integrating a sync feature with a smartwatch platforms such as \textit{Android Wear}\footnote{\url{https://www.android.com/intl/en_uk/wear/}} would enable the user to self-manage their activity levels even without a smartphone. That would be particularly useful because people who wear watches are more likely to use them most of the time which would lead to more accurate measurements. Furthermore, optimising the system in terms of battery-consumption would be another future improvement. That could be done by experimenting with lower sampling rates of the accelerometer sensor. The user feedback gathered by the survey also suggests improving the \textit{History} screen \gls{ui} so that it is easier to understand i.e. the connection between the legend section at the top of the screen is confusing and it is not that obvious it is associated to the progress bars data representation. That could be improved by adding labels in each of the progress bars so it is clear which one shows the \gls{pa} and which one shows the \gls{mci} progress.

\section{Reflection}
In general, I am satisfied with the effort I have put into the project research and development. I have worked to the best of my abilities to satisfy the (BSc) Honours Computing (Graphics and Animation) degree requirements. As far as the project time management is concerned, I have spent approximately two-thirds of my time on the design and research stage. It allowed me to gather all of the required background information in order to complete the software implementation of the project. Also, the system design ensured that the mobile application implements the specifications required by a typical self-management system and it is feasible to implement given the time constraint of the project.

This undertaking allowed me to apply the university knowledge gathered during my four years of study. Also, I was able to practice and improve my industry experience gained with xDesign. Going into the project, I was interested in evaluating how effective mobile technology is in regards to self-managing people's active and sedentary time. Now that the project is completed, I feel confident that mobile applications such as \textit{ActiveMinutes} can be used to help people improve their lifestyle.

\section{Conclusion}
This project produced a system that implements self-management approach to reduce sedentary behaviour. The users of the system can self-manage \gls{pa} and \gls{sb}. That has been achieved by implementing a goal-setting component allowing the user to change their goals; a monitoring component which monitors the user's activity levels and a feedback component which provides goal-performance notifications to the user for both active and inactive time in the form of notifications. The evaluation of the system suggested that it works as intended and successfully applies behaviour change. The survey participants were positively affected by using the system, i.e.\ tried to reduce prolonged inactivity intervals and introduce physical activity breaks doing walking, cycling or running.