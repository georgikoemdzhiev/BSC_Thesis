\chapter{Conclusion}
To sum up, the project initially set seven objectives, namely to research the effect of \gls{sb} and behaviour change approaches; research \gls{har}; develop a working \gls{har} system; integrate the latter with a \gls{sb} self-management system; design a mobile application that applies the research findings; evaluate and release the application. By achieving these project objectives the high-level project aim has beem achieved. That is to develop a system that allows for the self-management of \gls{sb} and \gls{pa}. In addition, the system provides the user with goal-performance feedback. 

The purpose of this chapter is to present the main findings discovered during the project development. Furthermore, feature versions of the mobile application will be discussed as well as discuss my experience throughout the duration of the project. 

\section{\textit{ActiveMinutes}}
As it was found in the literature review chapter, most of the current activity trackers focus more on monitoring people's activity levels rather than how much time they spend sedentary, so more research is needed into the realm of self-management mobile application that monitor both \gls{pa} and \gls{sb}.

This project produced a system that applies self-management for both physical states, namely \textit{Active} and \textit{Static}. It successfully applies behaviour change via three stages - goal-setting, monitoring and providing user feedback. The system was evaluated by real users and it was concluded that it works as expected.

\section{Feature versions}
Although \textit{ActiveMinutes} implements the key requirements needed by a self-management system (e.g. goal-setting, monitoring and user feedback) there are features that can be added in future iterations of the system. 

One of the features that may be addressed in future versions of the system is classifier personalisation. When implemented, it is expected that the system would improve its classification accuracy since the classifier would adapt to the user's specific motor patterns (i.e. walking). Furthermore, personalisation could be applied to the notification content. For example, the system may keep track of what messages have proven to be successfully (e.g. made the user more active) so that only the selected message content is shown to the user.

One of the features that was out of the scope if this project was social media sharing. Future versions of the application could incorporate a sharing feature allowing for sharing content on social media platforms such as \textit{Facebook} and \textit{Twitter} such as the set \gls{pa} goal. As it was found in the literature review (see section \ref{subsection:goal-moderators}) making a goal public increases how important is the goal in ones eyes and that is expected to produce better goal-performance results. 

The feedback gathered via the survey discussed in Chapter \ref{chapter:testing-and-eval} helped to identify other future improvements to the system. For example, integrating a sync feature with a smartwatch platform such as \textit{Android Wear}\footnote{\url{https://www.android.com/intl/en_uk/wear/}} would enable the user to self-manage their activity levels even without a smartphone. That would be particularly more useful when doing activities such as swimming. Furthermore, optimising the system in terms of battery-consumption would be another future improvement. That could be done by experimenting with lower sampling rates of the accelerometer sensor. The user feedback gathered by the survey also suggests improving the \textit{History} screen \gls{ui} so that it is easier to understand what each of the two progress bars mean. That could be improved by adding labels in the progress bars themselves so it is clearer which one shows \gls{pa} and which one shows \gls{mci} progress.

\section{Reflection}
In general, I am satisfied with the project development and the effort I have put into the project research and development. I have worked to the best of my abilities to satisfy the (BSc) Honours Computing (Graphics and Animation) degree requirements.

As far as the project time management is concerned, I have spent approximately two-thirds of my time on the design and research stage. Firstly, research allowed me to gather all of the required background information in order to complete the software implementation of the project. Secondly, the mobile application design ensured that the system implements the core specifications required by a typical self-management system and at the same time make sure that the system is feasible to implement given the project time constraint.

This project allowed me to apply the university knowledge gathered during my four years of study. Also, I was able to practice and improve my industry mobile application development experience gained with xDesign. Going into the project, I was curious about mobile development and how it can be used to help people in their everyday life. Now that the project is complete, I feel confident that mobile applications can certainly be used to support people, more specifically, to help inactive people become more active via the use of mobile systems.
