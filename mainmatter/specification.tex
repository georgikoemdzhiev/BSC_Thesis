\chapter{Specification}
\label{Chapter:Specification}

The specification of the mobile self-management application can be split into three components: Goal-setting, Monitoring and Feedback. The Goal-Setting Component (GSC), as the name implies, handles setting goals as well as providing additional information such as recommended amounts of physical activity per day. The Monitoring Component (MC) is the most complex part of the system. It is essentially a \gls{har} system with additional logic for recognising the current activity of the user (e.g. walking or static) and logging the data into a database. In addition, this component is responsible for detecting sedentary behaviour and measuring physical activity amounts. As far as the Feedback Component (FC) is concerned, it provides feedback to the user in the form of notifications. 
\section{Functional Requirements}

    \subsection{Goal-Setting Component}
    As it was mentioned in Chapter \ref{Chapter:Background} the mobile application proposed in this work is designed for people who are spending prolonged intervals of time in a static position during their day. This component's requirements are mainly derived from the background research. 
    
    \begin{enumerate}
        \item \gls{gsc} shall allow the user to set \gls{pa} goals
        \begin{enumerate}
            \item \gls{gsc} shall provide list of recommended \gls{pa} goals
            \item \gls{gsc} shall allow the user to set custom \gls{pa} goals
        \end{enumerate}
        \item \gls{gsc} shall allow the user to set \gls{st} goals
    \end{enumerate}
    
    
    \subsection{Monitoring omponent}
    The main purpose of the HAR component is to continuously classify human activities. The following requirements have been derived from analysing past and current human activity recognition mobile applications and devices.
    
    \begin{enumerate}
        \item The HAR component shall recognise human activities
        \begin{enumerate}
            \item HAR component shall recognise the following activities:
            \begin{enumerate}
                \item walking
                \item running
                \item static
                \item on a vehicle
                \item ascending stairs
                \item descending stairs
            \end{enumerate}
        \end{enumerate}
    \end{enumerate}
    
    \subsection{Feedback Component}
    This section lists functional requirements that are part of the the general mobile application logic.
    \begin{enumerate}
        \item The system shall provide user authentication
        \item The system shall store the recognised activities on the device
        \item The system shall display a list of past activities (e.g., daily and weekly organised)
    \end{enumerate}

\section{Non-functional Requirements}
The mobile application proposed in this work has to comply to a number of additional requirements. Since the components mentioned above are part of the same system their specifications will be discussed collectively.

    \subsection{Limitations}
    to be done
    
    \subsection{Accuracy}
    to be done
    
    \subsection{Power consumption}
    to be done 
    
    \subsection{Security Requirements}
    The proposed system in this work will not store or share user data outside the system. In addition, to ensure that the user's data is protected, the system will encrypt the stored user data (i.e. log-in credentials). 
    
    \subsection{System Quality Attributes}
    Introduction to this section here.
        \subsubsection{Adaptability}
        to be done
        \subsubsection{Scalability}
        to be done
        \subsubsection{Testability}
        to be done
        \subsubsection{Maintainability}
        to be done
    
    
    