\chapter{Project Considerations}
\label{chapter:project-considerations}
This chapter addresses the relevant professional, social, ethical, security and legal issues taken into consideration throughout the project development. As a student studying Computing for Graphics and Animations – a course that is accredited by the British Computer Society (BCS) I comply with the BCS Code of Conduct \citep{bcs_2017}. The following sections will discuss how this is achieved.
    
    \section{Professional}
    According to point 4 (d) of the BCS code of conduct an individual should “act with integrity and respect in your professional relationships with all members of BCS…”. Thus, I acknowledge the work of all external software authors used in this work (see Appendix \ref{chapter:third-party-software}).

    \section{Social}
    Point 3 (b) is especially important for the nature of the project. It states that an individual should "avoid any situation that may give rise to a conflict of interest...". The proposed application avoids promoting any body image stereotypes by focusing more on encouraging the users to be more active rather than on their body characteristics such as weight and height or BMI (Body Mass Index). In addition, the system only shows the recommended levels of physical activity and will not force the users of the application to follow those strictly (some jobs do require seating for extended intervals of time – e.g. drivers). That has been achieved by proving the option to in the \textit{Settings} screen to disable inactivity notifications when necessary.
    
    \section{Ethical}
    Since the development of the project required data collection (i.e. recording accelerometer data) involving different people, each participant has been given a consent form to sign. The form ensures that the project participants know the purpose of the project and more importantly the use of their data. The consent forms can be seen in Appendix \ref{chapter:consent-forms}. Furthermore, the data collected during the project development such as the names of the project participants and their accelerometer sensor data are used only for the development purposes of the project. 
    
    \section{Security}
    As the author if this work I intend to respect and comply with point 1 (a) of the conduct that states "you shall have due regard for public health, privacy, security...". Thus, the developed system implements an authorisation component to prevent other people accessing personal data such as total minutes of activity/inactivity collected daily. Furthermore, any relevant or valuable user data is anonymised to protect individual’s identity. For example, the system does not use nor asks the user for their name in order to operate i.e. user is only required to create \textit{username} which does not need to be the same as their actual name.

    
    \section{Legal}
    To attempt preventing physical harm to the user, the system shows a disclaimer (warning) dialog message to inform the user that they should be in a good physical condition i.e. not suffering from any diseases that could be worsened by using the mobile application. In addition, I intend to fully comply with the terms of conditions of any third party software that I intend to use. As far as the project participant’s data is concerned, I intend to keep the data only for the purposes of this project, and not use it for any commercial purpose.