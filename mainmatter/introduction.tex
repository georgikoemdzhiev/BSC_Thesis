\chapter{Introduction}
\label{Chapter:Introduction}
Nowadays, with the rapid advance of technology people spend a lot of their time in a static position. According to \citet{wilmot2012} that increases the chances of all-cause mortality by 49\%.  Also, people who lead non-active lifestyle are 112\% more likely to get diabetes, 147\% more likely to experience cardiovascular events, and 90\% more likely to die due to cardiovascular events. 
    
Using self-manageing systems to counter sedentary behaviour is becoming a well-accepted solution due to its many advantages over the traditional paper based methods. This paper proposes a self-management mobile application that apples behaviour change logic to reduce sedentary time as well encourage more physical activity. 
    
    \section{Project Catalyst}
    My passion for software began back in my first year as a student at Robert Gordon University. I was exposed to various modules related to software development which I found absorbing. Consequently, I was able to do a summer internship with a company called xDesign as a Software Engineer and thus gain industry knowledge and skills. I was confident that mobile application development was an area I wanted to pursue. 
    
    This passion for software contributed immensely on the topic of my honours project. Also, the idea of the health aspect of the mobile application made the subject even more appealing to me. My application could potentially help people self-manage their sedentary time as well as promote more physical activity. 
    
    
    \section{Goals and Objectives}
    A set of goals and objectives were formed prior to the initiation of the project to help shape its scope. The goals of the project define the main motivation behind it whereas the objectives underline the main tasks that the project undertakes.
    
    \subsection*{Goals}
    The following project goals have been identified: 
    \begin{enumerate}
        \item Satisfy the requirements the of BSc (Hons) Computing (Graphics and Animation) course
        \item Demonstrate future employers programing as well as UI/UX design skills
        \item Gain a better understanding of how machine learning works 
        \item Go through the main stages of the software development process
    \end{enumerate}
    
    \subsection*{Objectives}
    The following list of objectives helped accomplishing the above goals: 
    \begin{enumerate}
        \item Gather information about Self-Management in the Health sector
        \item Research Sedentary Behaviour factors and consequences
        \item Research behaviour change methods
        \item Research how to implement Human Activity Recognition 
        \item Design and Implement a mobile application on the Android platform based on the project research findings
    \end{enumerate}
    
    \section{Document structure}
    This section of the document addresses how this report is organised.\newline
    
    \textbf{Literature Review} Gives a wide background information regarding using self-management technologies in the health sector, Behaviour changing techniques, consequences and current situation of of Sedentary Behaviour as well as researching commercial and non-commercial HAR mobile applications.\newline
    
    
    \textbf{Specification} Lists the proposed mobile application Functional and Non-functional requirements as well as discussing the system quality attributes. The information in this chapter is heavily influenced by the \textit{Literature Review} chapter.\newline
    
    
    \textbf{Design} Addresses the system design decisions\newline
    
    
    \textbf{Implementation} Details about how the system design have been implemented for each of the main system components.\newline
    
    
    \textbf{Evaluation} A discussing how effective is the implemented system on users.\newline
    
    
    \textbf{Conclusion} Summary of the potential benefits of Sedentary Behaviour self-manageing devices in the health sector, future work and project critical appraisal.\newline
    
    
    \section{Professional, Social, Ethical, Security and legal issues to the project}
    
    \subsection{Professional}
    As a student studying Computing for Graphics and Animations – a course that is accredited by the British Computer Society (BCS) I intend to comply with the BCS Code of Conduct. Especially, point 4 (d) \citep{bcs_2017} - “act with integrity and respect in your professional relationships with all members of BCS…”. Thus, I intend to acknowledge the work of all external software authors used in this work (see Appendix \ref{chapter:third-party-software}).

    \subsection{Social}
    The proposed application will try to avoid promoting any body image stereotypes by focusing more on encouraging the users to be more active rather than on their body characteristics such as weight and height or BMI (Body Mass Index). In addition, the application will only show what the recommended activity intervals per day are and will not force the users of the application to follow those strictly (some jobs do require seating for extended intervals of time – e.g. Pilots).
    
    \subsection{Ethical}
    The data collected during the project development such as names of the project participants and their accelerometer sensor data will be stored on disk only for the development purposes of the project. 
    
    \subsection{Security}
    The mobile application, proposed in this work, will offer an authorisation component to prevent other people accessing personal data such as total minutes of activity/inactivity collected daily. What is more, any relevant or valuable user data will be anonymised to protect individual’s identity.

    
    \subsection{Legal}
    To prevent any harm done to the user, the mobile application will show a disclaimer (warning) dialog message to inform the user that they should be in a good physical condition (not suffering from diseases that could lead to worsening the condition of the user) before they use the application. In addition, I intend to fully comply with the terms of conditions of any third party software that I intend to use (my main goal is to use primarily Open Source software). As far as the project participant’s data is concerned, I intend to keep the data only for the purposes of this project, and the data will not be used for any commercial purpose. 
