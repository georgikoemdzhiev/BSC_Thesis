\chapter{Introduction}
\label{Chapter:Introduction}
Nowadays, with the rapid advance of technology people spend a lot of their time in a static position. According to \citet{wilmot2012} that increases the chances of all-cause mortality by 49\%. Also, people who lead non-active lifestyle are 112\% more likely to get diabetes, 147\% more likely to experience cardiovascular events, and 90\% more likely to die due to cardiovascular events. According to \citet{cerb_2015}, sedentary behaviour costs the EU economy 80 billion euros annually and is found to be the leading cause of absence from work \citep{Wynne-Jonesoemed_2013}.
    
Using self-management systems such as \textit{Selfback} \citep{selfback_2016} to counter sedentary behaviour is becoming a well-accepted solution due to its many advantages over the traditional paper-based methods. This paper proposes a mobile application that aims to help inactive people reduce sedentary time by implementing self-management logic. It allows the user to set physical activity and sedentary time goals and provides them with goal-performance feedback in the form of notifications i.e.\ when a goal is achieved and when extended inactivity is detected. The system monitors physical activity and sedentary time in real-time by incorporating a Human Activity Recognition (\gls{har}) system capable of recognising four activities i.e. walking, running, cycling and static.
    
    %\section{Project Catalyst}
    %My passion for software began back in my first year as a student at Robert Gordon University. I was exposed to various modules related to software development which I found absorbing. Consequently, I was able to do a summer internship with a company called xDesign as a Software Engineer and thus gain industry knowledge and skills. I was confident that mobile application development was an area I wanted to pursue. 
    
    %This passion for software contributed immensely on the topic of my honours project. Also, the idea of the health aspect of the mobile application made the subject even more appealing to me. My application could potentially help people self-manage their sedentary time as well as promote more physical activity using only the build-in accelerometer sensor on a smartphone. 
    
    
    \section{Aim and Objectives}
    %An aim and objectives were formed prior to the initiation of the project to help shape its scope. The aim of the project define the main motivation behind it whereas the objectives underline the main tasks that the project undertakes.
    
    \subsection*{Aim}
    The aim of this project is to develop a mobile application to support people with inactive lifestyles as they attempt to change their behaviour and become more active. The behaviour change will be achieved by building a self-management system that allows the user to set their goals. The built-in sensors of the mobile device will then be used to monitor the user's progress towards the set goals with achievement-aware notifications being pushed to the user to provide motivation.
    
    \subsection*{Objectives}
    The following list of objectives helped to accomplish the above goals: 
    \begin{enumerate}
        \item Research the effect of Sedentary Behaviour (\gls{sb}) as well as investigate behaviour change approaches
        \item Investigate and gather information about Human Activity Recognition (\gls{har}) on wearable devices
        \item Develop a fully working \gls{har} system
        \begin{itemize}
            \item Gather training data
            \item Use the above to train a classifier (needed to recognise user activities such as walking)  
        \end{itemize}
        \item Incorporate the above within a \gls{sb} self-management system
        \begin{itemize}
            \item Allows the user to set goals
            \item Perform user activity monitoring
            \item Provides feedback to the user
        \end{itemize}
        \item  Design and Implement a mobile application based on the project research findings
        \begin{itemize}
            \item Produce a system architectural and \gls{ui} design
            \item Incorporate the research findings in the system as business logic 
        \end{itemize}
        \item Evaluate the system performance using user feedback and analysing the gathered data
        \begin{itemize}
            \item Test the system using varies techniques such as Acceptance, Functional and Usability tests
            \item Gather user feedback regarding the overall functionality of the system via survey
        \end{itemize}
        \item Release app to application to the Application Store 
        \begin{itemize}
            \item Publish the mobile application to the general public
        \end{itemize}
    \end{enumerate}
    
    \section{Document structure}
    This section of the document addresses how this report is organised.
    \begin{itemize}
        \item \textbf{\ref{Chapter:Literature-Review}.\ Literature Review} Gives a wide background information regarding using self-management technologies in the health sector, Behaviour changing techniques, consequences and current situation of of Sedentary Behaviour as well as researching commercial and non-commercial HAR mobile applications.
        \item \textbf{\ref{Chapter:Specification}.\ Specification} Lists the proposed mobile application Functional and Non-functional requirements as well as discussing the system quality attributes. The information in this chapter is heavily influenced by the \textit{Literature Review} chapter.
        \item \textbf{\ref{Chapter:Design}.\ Design} Addresses the system architectural and \gls{ui} design decisions.
        \item \textbf{\ref{chapter:implementation}.\ Implementation} Details about how the system design have been implemented for each of the main system components.
        \item \textbf{\ref{chapter:testing-and-eval}.\ Testing and Evaluation} Analysing the system in terms of Acceptance, Usability and Functional testing. Also, evaluating the effectiveness of the system on users.
        \item \textbf{\ref{chapter:project-considerations}.\ Project Considerations} A discussion on the professional, social, ethical, security and legal issues relevant to the project.
        \item \textbf{\ref{chapter:conclusion}.\ Conclusion} Summary of the potential benefits of Sedentary Behaviour self-manageing devices in the health sector, future work and project critical appraisal.
    \end{itemize}
