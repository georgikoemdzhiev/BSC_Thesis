\chapter{Implementation}
This chapter discusses the implementation details of the proposed mobile application. First, the \gls{har} stages will be discussed. That includes the data gathering from the projects' participants as well as data prepossessing and training a classifier. Next, the specifics of the mobile application itself will be discussed such as database and overall system design implementation.


\section{Human Activity Recognition}
The application needs to recognises human activities (i.e. walking and running) in order for self-management process to work. A human activity recognition system is needed to periodically recognise the activities of the smartphone user so the self-management system can take actions such as notify the person that they are being sedentary for too long. Implementation of the \gls{har} system will be discussed bellow.

    \subsection{Data Collection}
    One of the key stages in the development process of a typical supervised (online) \gls{har} system is the data collection stage. The data was collected from 3 fellow students in a controlled environment. All of the participants recorded 3 minutes of each one of the following activities: (i.e. \textit{walking},\textit{running},\textit{cycling} and \textit{static}. The device used to collect the data was \textit{OnePlus One}. 
    
    Android Operating System provides a total of four different sampling frequencies for reading the accelerometer sensor, namely \textit{NORMAL: 5 Hz}, \textit{UI: 16 Hz}, \textit{GAME: 50 Hz}, and \textit{FASTEST}. The latter has been chosen for the sampling frequency (SF) of the build-in accelerometer as it has produces good results in \citet[3-5]{lee2016}. A dataset containing ~720 labeled data points was produces as a result of the data collection prices (3 second time window x 3 participants x 4 types of activities).