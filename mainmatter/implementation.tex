\chapter{Implementation}
This chapter discusses the implementation details of the proposed mobile application. First, the \gls{har} stages will be discussed. That includes the data gathering from the projects' participants as well as data prepossessing and training a classifier. Next, the specifics of the mobile application itself will be discussed such as database and overall system design implementation.

\section{The mobile application}

    \subsection{MVP and Dagger Dependency Injection framework}
    
    \subsection{Realm Database}
    
    \subsection{Data collection service}
    
    \subsection{Active Minutes service}
    


\section{Human Activity Recognition}
The application needs to recognises human activities (i.e. walking and running) in order for self-management process to work. A human activity recognition system is needed to periodically recognise the activities of the smartphone user so the self-management system can take actions such as notify the person that they are being sedentary for too long. Implementation of the \gls{har} system will be discussed bellow.

    \subsection{Data Collection}
    One of the key stages in the development process of a typical supervised (online) \gls{har} system is the data collection stage. A good quality data is needed to train a model (or also called classifier) by providing it with labelled data. Consequently, this model can classify an unseen (or unlabelled) data to a specific activity (i.e. walking or being static).
    
    The data needed to train the model in this work was collected from 3 fellow students in a controlled environment. Each one of the project participants was equipped with a device running partially implemented \textit{"Active Minutes"} application (e.g. only the \textit{"Data Collection screen"} see \ref{fig:data-collection-screen-design}) into their pants pocket. This location has been found to be the optimal position for \gls{har} (see \ref{section:non-commercial-har-systems}). All of the participants recorded 3 minutes of each one of the following activities: \textit{walking}, \textit{running}, \textit{cycling} and \textit{static}. They had to press the start "START" to start recording process on the device. They performed an activity for 3 minutes and then pressed the "STOP" button to stop the recording.
    
    To avoid any unwanted information during the recording stage, the first and the last 7 seconds of each activity was removed since it contained noise information such as the linear acceleration taking place when the device was put in and pulled out of participants pockets. A dataset containing ~720 labelled data points (or a total of ~36 minutes of data) was produced as a result of this process. 
    
    When all of the required data was collected it was converted into a WEKA ARFF file format by pressing the "EXPORT" button on the Data Collection screen. The ARFF file was stored on the device external memory. A sample of the produced ARFF file can be seen in Listing \ref{weka-arff-code}.
    
\begin{lstlisting}[caption=WEKA ARFF file extract,
label=weka-arff-code,captionpos=b, frame=single,basicstyle=\small,float,floatplacement=H,breaklines=true]
@relation HAR
    
@attribute accX__fft1 numeric
@attribute accX__fft2 numeric
@attribute accX__fft3 numeric
@attribute accX__fft4 numeric
@attribute accX__fft5 numeric
@attribute accY__fft1 numeric
@attribute accY__fft2 numeric
@attribute accY__fft3 numeric
@attribute accY__fft4 numeric
@attribute accY__fft5 numeric
@attribute accZ__fft1 numeric
@attribute accZ__fft2 numeric
@attribute accZ__fft3 numeric
@attribute accZ__fft4 numeric
@attribute accZ__fft5 numeric
@attribute accM__fft1 numeric
@attribute accM__fft2 numeric
@attribute accM__fft3 numeric
@attribute accM__fft4 numeric
@attribute accM__fft5 numeric
@attribute class {walking,running,static,cycling}
    
@data
9,-76,-834,25,22,4,-1,-401,-190,-3,3,5,-213,-94,-45,2,-50,-49,-15,18,walking
-0,-184,140,4,-34,-2,13,45,63,-8,0,51,-68,-30,-72,1,-52,15,-51,63,walking
-1,-13,-30,3,2,3,-18,-2,3,-21,-1,2,-4,-9,-1,-2,33,3,16,20,walking
0,-16,-17,-22,6,-3,-6,-8,-16,-3,-15,-2,12,-5,9,8,8,37,38,-18,walking
-6,-22,14,-54,1,14,8,27,9,60,-22,-31,-11,-11,4,-7,1,0,-3,-39,running
-3,-11,-35,-3,-53,-13,-45,23,-57,17,-2,52,13,58,15,22,48,-36,25,-50,running
3,7,-11,27,-41,-19,-31,-53,-107,10,0,-42,22,-42,3,20,34,-26,76,-104,running
-3,37,34,9,58,0,30,97,-74,83,4,-13,-39,40,26,-4,15,-10,27,-26,cycling
-1,-91,-131,-121,-58,-4,6,-66,22,70,3,-30,-76,-63,-30,-1,-39,-10,-39,-76,cycling
1,-54,-41,-180,114,-3,4,-21,78,-188,3,8,-56,-19,70,0,-38,22,-83,103,cycling
0.3,0,-1,1,-1.4,0,1,-1,0,0,0,-1,0,0,0,0,0,0,0,0,static
0,-6,48,-8,0,0,-6,-21,2,5,0,-1,42,-6,0,0,0,-1.5,0,-1.6,static
-0.3,8,-4,3,-4,0.1,6,-4,2,-3,0,2,-2,2,-1,0,0.2,1,-0.4,0.2,static
\end{lstlisting}
    
    The device used to collect the data was \textit{OnePlus One}. Android Operating System provides a total of four different sampling frequencies for reading the accelerometer sensor, namely \textit{NORMAL: 5 Hz}, \textit{UI: 16 Hz}, \textit{GAME: 50 Hz}, and \textit{FASTEST}. The latter has been chosen for the sampling frequency (SF) of the built-in accelerometer as it has produced good results in \citet[3-5]{lee2016}. According to the Google's documentation \textit{FASTEST} SF depends on the hardware of the device \citep{googlesensormanager2017}. In this case, the SF was \textbf{115} Hz. 
    
    \subsection{Training the classifier}
    THE APPLICATION CREATES THE CLASSIFIER UPON THE FIRST LAUNCH
    
    
    