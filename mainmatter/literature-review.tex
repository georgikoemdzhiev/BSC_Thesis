\chapter{Literature Review}
\label{Chapter:Literature-Review}

As it was mentioned in Chapter \ref{Chapter:Introduction} this document proposes a system implemented on a mobile device that encourages the reduction of sedentary time via self-management techniques. In order to gain knowledge on how to devise such as system, literature review needs to be carried away on several topics. First of all, the current shift towards self-management in the health sector will be discussed. The next section will focus on evaluating how much time people spend in Sedentary Behaviour (SB) and what are the consequences. Section 3 looks at digital behaviour change techniques. Commercial and non-commercial \gls{har} mobile applications and \gls{har} itself are discussed in Section 4. Section 5 summarises the system proposed.

\section{Self-Management in the Health Sector}

Mobile devices with embedded self-management logic are currently being utilised and becoming a well-accepted solution for millions of people in the health sector. For example, the NHS’s England Executive – Simon Stevens, launched a programme “Test Beds” \citep{nhsengland2016,nhsengland2016a} which is a set of collaborative projects between NHS and some technology companies such as Verily, IBM and Philips. The idea behind the project is to test the effectiveness of different technological innovations, including wearable devices and mobile applications. These technologies will enable patients to self-manage illnesses such as diabetes, heart diseases and dementia.